% !TEX program = xetex
\PassOptionsToPackage{usenames,dvipsnames}{xcolor}
\PassOptionsToPackage{colorlinks,linktoc=all}{hyperref}
% \documentclass[sigconf, anonymous, review]{acmart} % for review
\documentclass[sigconf]{acmart} % for arxiv

%%
%% \BibTeX command to typeset BibTeX logo in the docs
\AtBeginDocument{%
  \providecommand\BibTeX{{%
    \normalfont B\kern-0.5em{\scshape i\kern-0.25em b}\kern-0.8em\TeX}}}

%% Rights management information.  This information is sent to you
%% when you complete the rights form.  These commands have SAMPLE
%% values in them; it is your responsibility as an author to replace
%% the commands and values with those provided to you when you
%% complete the rights form.
\setcopyright{acmcopyright}
\copyrightyear{2018}
\acmYear{2018}
\acmDOI{10.1145/1122445.1122456}

%% These commands are for a PROCEEDINGS abstract or paper.
\acmConference[KDD 2019]{}{August 4-8, 2019}{Anchorage, Alaska, USA}
\acmBooktitle{Woodstock '18: ACM Symposium on Neural Gaze Detection,
  June 03--05, 2018, Woodstock, NY}
\acmPrice{15.00}
\acmISBN{978-1-4503-XXXX-X/18/06}

%%
%% Submission ID.
%% Use this when submitting an article to a sponsored event. You'll
%% receive a unique submission ID from the organizers
%% of the event, and this ID should be used as the parameter to this command.
%%\acmSubmissionID{123-A56-BU3}

%%
%% The majority of ACM publications use numbered citations and
%% references.  The command \citestyle{authoryear} switches to the
%% "author year" style.
%%
%% If you are preparing content for an event
%% sponsored by ACM SIGGRAPH, you must use the "author year" style of
%% citations and references.
%% Uncommenting
%% the next command will enable that style.
%%\citestyle{acmauthoryear}

% !TEX root = ../recommending-interesting-writing.tex

% FONTS
%\usepackage[T1]{fontenc}

% Replace default Latin Modern typewriter with its proportional counterpart
% http://www.tug.dk/FontCatalogue/lmoderntypewriterprop/
%\renewcommand*\ttdefault{lmvtt}

%%% OPTION 3 - MTPRO 2 Math + Termes Times + ParaType Sans
\let\myBbbk\Bbbk
\let\Bbbk\relax
\let\bibsep\relax
\let\openbox\relax
\let\proof\relax
\let\endproof\relax
%\usepackage{tgtermes}
\usepackage{minted}
\usepackage{amsmath}
% \usepackage[subscriptcorrection,
%             amssymbols,
%             mtpbb,
%             mtpcal,
%             nofontinfo  % suppresses all warnings
%            ]{mtpro2}
% \usepackage{scalefnt,letltxmacro}
% \LetLtxMacro{\oldtextsc}{\textsc}
% \renewcommand{\textsc}[1]{\oldtextsc{\scalefont{1.10}#1}}
% \usepackage[scaled=0.92]{PTSans}

% ICONS
\usepackage{fontawesome}

% CODE

% COLOR
\usepackage[usenames,dvipsnames]{xcolor}
\definecolor{shadecolor}{gray}{0.9}

% SPACING and TEXT
%\usepackage[final,expansion=alltext]{microtype}
%\usepackage[english]{babel}
\usepackage[parfill]{parskip}
\usepackage{afterpage}
\usepackage{framed}
\usepackage{nicefrac}

% EDITING
% line numbering in left margin
\usepackage{lineno}
\renewcommand\linenumberfont{\normalfont
                             \footnotesize
                             \sffamily
                             \color{SkyBlue}}
% ragged paragraphs in right margin
\usepackage{ragged2e}
\DeclareRobustCommand{\sidenote}[1]{\marginpar{
                                    \RaggedRight
                                    \textcolor{Plum}{\textsf{#1}}}}

% Define a paragraph header function
\DeclareRobustCommand{\parhead}[1]{\textbf{#1}~}

% paragraph helper
\DeclareRobustCommand{\PP}{\textcolor{Plum}{\P}~}
\DeclareRobustCommand{\pp}{\textcolor{Plum}{\P}~}

% COUNTERS
\renewcommand{\labelenumi}{\color{black!67}{\arabic{enumi}.}}
\renewcommand{\labelenumii}{{\color{black!67}(\alph{enumii})}}
\renewcommand{\labelitemi}{{\color{black!67}\textbullet}}

% FIGURES
\usepackage{graphicx}
\usepackage[labelfont=bf]{caption}
\usepackage[format=hang]{subcaption}

% TABLES
\usepackage{booktabs}
%\usepackage{dblfloatfix}  % for placing table at bottom of page

% TABLE ALIGNMENT
\usepackage{etoolbox,siunitx}
\robustify\bfseries
\sisetup{detect-weight=true, detect-shape=true, detect-mode=true,
table-format=5.1,
table-number-alignment=center,
separate-uncertainty=true,
input-ignore={,},input-decimal-markers={.}}

% BABEL
% \usepackage{polyglossia}

% BIBLIOGRAPHY
\usepackage[backend=biber, style=numeric-comp]{biblatex}
%\usepackage{natbib}

% ALGORITHMS
\usepackage[algoruled]{algorithm2e}
\usepackage{listings}
\usepackage{fancyvrb}
\fvset{fontsize=\normalsize}

% THEOREMS
\usepackage{amsthm}
\newtheorem{theorem}{Theorem}
% \newtheorem{proposition}[proposition]{Proposition}
\newtheorem{prop}{Proposition}


% TODO
%\usepackage{todo}

% HYPERREF
%\usepackage[colorlinks,linktoc=all]{hyperref}
\usepackage[all]{hypcap}
\hypersetup{citecolor=Violet}
\hypersetup{linkcolor=black}
\hypersetup{urlcolor=MidnightBlue}

% CLEVEREF must come after HYPERREF
\usepackage[capitalize]{cleveref}

% ACRONYMS
\usepackage[acronym,smallcaps,nowarn]{glossaries}
% \makeglossaries

% COLOR DEFINITIONS
\newcommand{\red}[1]{\textcolor{BrickRed}{#1}}
\newcommand{\orange}[1]{\textcolor{BurntOrange}{#1}}
\newcommand{\green}[1]{\textcolor{OliveGreen}{#1}}
\newcommand{\blue}[1]{\textcolor{MidnightBlue}{#1}}
\newcommand{\gray}[1]{\textcolor{black!60}{#1}}

% LISTINGS DEFINTIONS
\usepackage{listings}
\lstdefinestyle{alp_style}{
    commentstyle=\color{OliveGreen},
    numberstyle=\tiny\color{black!60},
    stringstyle=\color{BrickRed},
    basicstyle=\ttfamily\scriptsize,
    breakatwhitespace=false,
    breaklines=true,
    captionpos=b,
    keepspaces=true,
    numbers=none,
    numbersep=5pt,
    showspaces=false,
    showstringspaces=false,
    showtabs=false,
    tabsize=2
}
\lstset{style=alp_style}

\input{preamble/preamble_math}
% !TEX root = set_recommendation.tex

\newacronym{ELBO}{elbo}{evidence lower bound}
\newacronym{GMM}{gmm}{Gaussian mixture model}
\newacronym{KL}{kl}{Kullback-Leibler}
\newacronym{LDA}{lda}{latent Dirichlet allocation}
\newacronym{SVI}{svi}{stochastic variational inference}
\newacronym{DEF}{def}{deep exponential family}
\newacronym{rfs}{rfs}{\textsc{rankfromsets}}
\newacronym{ctpf}{ctpf}{collaborative topic Poisson factorization}
\input{preamble/preamble_tikz}

\title{Ranking Feature Writing}
%\title{Recommending items using their sets of attributes}
% keywords:
% deep learning
% collaborative filtering
% negative sampling
% content-based recommendation
% food recommender systems
% \begin{framed}
% \centering
% \textsc{do not cite or redistribute.}
% \end{framed}

\author{Rohan Bansal}
\affiliation{%
  \institution{Central High School}
  % \streetaddress{1 Th{\o}rv{\"a}ld Circle}
  \city{Springfield, MO}
  \country{US}}
\email{rajeshr@cims.nyu.edu}

\author{Uri Bram}
\affiliation{%
  \institution{The Browser}
  % \streetaddress{1 Th{\o}rv{\"a}ld Circle}
  \city{Pallermo}
  \country{Argentina}}
\email{uri@thebrowser.com}

\author{Robert Cottrell}
\affiliation{%
  \institution{The Browser}
  % \streetaddress{1 Th{\o}rv{\"a}ld Circle}
  \city{Riga}
  \country{Latvia}}
\email{robert@thebrowser.com}

\author{Jaan Altosaar}
\orcid{0000-0003-1294-4159}
\affiliation{%
  \institution{Princeton University}
  % \streetaddress{1 Th{\o}rv{\"a}ld Circle}
  \city{Princeton}
  \country{United States}}
\email{altosaar@princeton.edu}

%%
%% By default, the full list of authors will be used in the page
%% headers. Often, this list is too long, and will overlap
%% other information printed in the page headers. This command allows
%% the author to define a more concise list
%% of authors' names for this purpose.
\renewcommand{\shortauthors}{Trovato and Tobin, et al.}

\begin{document}
% !TEX root = recommending-interesting-writing.tex
\begin{abstract}
  We build a visual interface for recommending articles to editors at The Browser, a curation service for interesting writing. From a large list of candidates, editors decide which articles are selected and shared with subscribers. To aid the editors in this decision-making task, we build a visual interface for a recommendation model, \gls{rfs}~\citep{altosaar2020rankfromsets:}, that classifies articles based on their words. Control of the recommendation model is built into the visual interface. For example, an editor can use a topic slider to receive a new list of recommendations according to topical words in articles. These topic sliders might be used to increase or decrease the ranking of articles with words related to crime, business, or technology. The visual interface is also designed to be explanation-aware: words that contribute positively or negatively to an article's ranking are displayed. For the backend of the visual interface, \gls{rfs} is trained on historical data. In an offline empirical study, we find that \gls{rfs} outperforms \acrshort{bert}~\citep{devlin2019bert:}, a competitive classification model, in terms of recall. Further, we measure \gls{rfs} to be 10 times faster to train and to return predictions 2000 times faster than \acrshort{bert}. This speed is a beneficial property for the visual interface, and we demonstrate that \gls{rfs} can be deployed on the free tier of AWS Lambda using a short python script and numpy dependency. For reproducibility, transparency, and trust of the visual interface, we open source and release a public demonstration,\footnote{\url{https://the-browser.github.io/recommending-interesting-writing/}\label{url:demo}} data collection, training and deployment scripts, and model parameters.\footnote{\url{https://github.com/the-browser/recommending-interesting-writing}}
\end{abstract}
%%
%% The code below is generated by the tool at http://dl.acm.org/ccs.cfm.
%% Please copy and paste the code instead of the example below.
%%
\begin{CCSXML}
<ccs2012>
 <concept>
  <concept_id>10010520.10010553.10010562</concept_id>
  <concept_desc>Computer systems organization~Embedded systems</concept_desc>
  <concept_significance>500</concept_significance>
 </concept>
 <concept>
  <concept_id>10010520.10010575.10010755</concept_id>
  <concept_desc>Computer systems organization~Redundancy</concept_desc>
  <concept_significance>300</concept_significance>
 </concept>
 <concept>
  <concept_id>10010520.10010553.10010554</concept_id>
  <concept_desc>Computer systems organization~Robotics</concept_desc>
  <concept_significance>100</concept_significance>
 </concept>
 <concept>
  <concept_id>10003033.10003083.10003095</concept_id>
  <concept_desc>Networks~Network reliability</concept_desc>
  <concept_significance>100</concept_significance>
 </concept>
</ccs2012>
\end{CCSXML}

\ccsdesc[500]{Computer systems organization~Embedded systems}
\ccsdesc[300]{Computer systems organization~Redundancy}
\ccsdesc{Computer systems organization~Robotics}
\ccsdesc[100]{Networks~Network reliability}

%%
%% Keywords. The author(s) should pick words that accurately describe
%% the work being presented. Separate the keywords with commas.
\keywords{datasets, neural networks, gaze detection, text tagging}

%% A "teaser" image appears between the author and affiliation
%% information and the body of the document, and typically spans the
%% page.
\begin{teaserfigure}
  \includegraphics[width=\textwidth]{sampleteaser}
  \caption{Seattle Mariners at Spring Training, 2010.}
  \Description{Enjoying the baseball game from the third-base
  seats. Ichiro Suzuki preparing to bat.}
  \label{fig:teaser}
\end{teaserfigure}

%%
%% This command processes the author and affiliation and title
%% information and builds the first part of the formatted document.
\maketitle
% !TEX root = recommending-interesting-writing.tex
\section{Introduction}
\label{sec:introduction}
Creative nonfiction, longform journalism, and blog posts are examples of the types of articles curated by The Browser's team of editors. The editors filter through a large number of articles from various publications to select content to recommend to subscribers.

Journalism and content curation services such as The Browser can be constrained by financial resources, and this affects the types of machine learning solutions that might be feasible to aid editors in filtering through large numbers of articles. For example, serving models on GPUs may be too expensive, and a trade-off must be made between cost and performance.

Similar to the trade-off in performance due to constraints, there is a trade-off in user privacy intrinsic to recommender systems. A machine learning model must be trained on historical consumption data, and for a performant model, the items in the training data should be similar to new content the recommender is designed to filter. However, the incentive structures of operating a recommender system within a business can influence decisions around privacy and transparency~\citep{diakopoulos2020oxford}. For example, advertising business models may engender recommender systems that upweight attention-grabbing content and hence time spent looking at ads. Such content might maximize a user's time spent with the service over time at the expense of long-term user experience or consent. Outside of recommender systems, privacy-preserving and open source tools such as the Signal encrypted messaging service\footnote{\url{https://signal.org/}} may provide improved user experience in terms of privacy-preserving, transparent, and explainable algorithms~\citep{cohn-gordon2017a-formal}.

We build an end-to-end recommender system to address two aims: (1) to recommend interesting writing to editors at The Browser that provides interpretable recommendations and (2) to release a lightweight, open-source machine learning framework for developing and deploying recommender systems for quality writing.
% !TEX root = set_recommendation.tex
\section{Background}
\newcommand{\x}{\faCheck}
\begin{table*}%[!b]
  \begin{center}
    \begin{tabular}{lcccccc}
      \toprule
      Recommendation model & Uses attributes & Only user-item interactions & Scalable & Permutation-invariant & Loss tied to evaluation \\
      \midrule
      \acrlong{rfs} & \x & \x & \x & \x & \x \\
      \citet{wang2011collaborative} & \x & \x & & \x & \\
      \citet{gopalan2014content-based}& \x & \x  &  & \x &\\
%      \citet{lian2018towards} & & & \x & & \\
      \citet{dong2017a-hybrid}& \x & & & \x &\\
      \citet{chen2017joint}&\x & &\x & &\\
      \citet{bansal2016ask-the-gru:}&\x & \x & & &\\
      \citet{xu2017tag-aware}& \x & & \x & \x &\\
      \citet{rendle2009bpr:}&  & \x &  & & \x\\
      \citet{shi2012tfmap:} &\x & \x & & \x & \x\\
      \citet{wu2018starspace:} &\x & &\x &\x &\\
      \citet{kula2015metadata} & \x & \x &  & \x &\\
      \citet{shi2012climf:} & & \x & & &\x \\
      \citet{chen2018a-collective} & \x & \x & & \x & \\
      \citet{liu2014recommending} & \x & & & \x & \x \\
      \citet{cao2017embedding} & \x & &  & & \x \\
      \citet{okura2017embedding-based} & \x & & \x & & \x \\
%      \citet{zhang2018discrete} & \x & \x & \x & \x &\\
      \bottomrule
    \end{tabular}
    \caption{\label{tab:background}\acrlong{rfs} is a scalable recommendation
      model that recommends items using attributes, and is trained on an
      objective function connected to an evaluation metric. Most methods we
      highlight use attributes; some require data in addition to the user-item
      matrix of observations. Some models are invariant to permutation of the
      attributes, and may use a loss function that is connected to a
      recommendation performance metric. Few methods are scalable, as most
      recommendation models that use item side information require learning
      parameters for every item. }
    \vspace{-0.5cm}
  \end{center}
\end{table*}


%%% Local Variables:
%%% mode: latex
%%% TeX-master: "../set_recommendation"
%%% End:


We highlight two themes in research on recommendation models. We describe
recommendation models that incorporate side information and models that optimize
proxies of ranking metrics, and summarize this related work
in~\Cref{tab:background}.

\paragraph{Recommendation with side information} Side information is included in
recommendation models in several ways; we focus on deep learning and matrix
factorization approaches. Item side information can be modeled with deep
representations
\cite{zhang2017deep,bansal2016ask-the-gru:,lian2018towards,dong2017a-hybrid,chen2017joint,liang2018trsdl:,zuo2016tag-aware,xu2017tag-aware}
or can be included in content-based matrix factorization models as an additional
matrix
\cite{shi2014collaborative,gopalan2014content-based,wang2011collaborative,zhen2009tagicofi:,loepp2019interactive,bogers2018tag-based}.
Some deep learning based approaches scale to large datasets, but may not have
loss functions tied to evaluation metrics, or require data besides user-item
interactions. Content-based matrix factorization methods require learning
parameters for every item, and do not scale to data with large numbers of items.

\paragraph{Learning to rank} Recommendation models can be trained on loss
functions that approximate ranking-based evaluation metrics
\cite{yu2018walkranker:,liang2018top-n-rank:,rendle2009bpr:,song2018neural}, and
these models may include side information
\cite{shi2012tfmap:,yuan2016optimizing,ying2016collaborative,cao2017embedding,okura2017embedding-based}.
Such approaches may require data in addition to the user-item matrix, per-item
parameters, or use models where the output depends on the ordering of item
attributes.

%%% Local Variables:
%%% mode: latex
%%% TeX-master: "set_recommendation"
%%% End:

% as they are relevant to
% modeling items with attributes

% \paragraph{Deep representations of side information}
% Deep learning-based recommendation models incorporate side information in
% multiple ways \cite{zhang2017deep}. For example, items that have words as
% attributes can be represented using recurrent neural
% networks~\cite{bansal2016ask-the-gru:}. \citet{lian2018towards} use an attention
% mechanism to weight recommendations according to available item and user side
% information but not attributes, and \citet{dong2017a-hybrid} use denoising
% autoencoders to model side information in a deep recommendation model, but
% requires fitting parameters for every item. An example of a more efficient
% approach is the method in \citet{chen2017joint}, where embeddings are jointly
% learned for users, items, and item text for recommendation, but this method
% focuses on unsupervised pre-training of text representations. There are several
% examples of `tag-aware' or `tag-based' deep recommendation
% models~\cite{liang2018trsdl:,zuo2016tag-aware}, such as
% \citet{xu2017tag-awarey}, which focuses on data where users and items have
% different attributes and use autoencoders to learn user, item, and attribute
% representations. They use a cosine similarity-based objective function which is
% not tied to a metric used to evaluate recommendation performance.
%, but this method requires
%collecting information about which items users decided not to consume.

% \paragraph{Matrix factorization with side information}
% \citet{shi2014collaborative} survey several matrix factorization methods that
% incorporate side information. \citet{gopalan2014content-based} develop a
% Bayesian matrix factorization model for recommending items based on side
% information in the form of words in documents. \citet{wang2011collaborative}
% develop a regression model that uses a topic model to incorporate side
% information into recommendations. There are also several `tag-based' or
% `tag-aware' content-based matrix factorization
% models~\cite{zhen2009tagicofi:,loepp2019interactive,bogers2018tag-based}. Such
% content-based matrix factorization methods maximize the conditional
% log-likelihood of the data (or a bound on the log-likelihood); optimizing these
% objective functions does not optimize an evaluation metric. Furthermore, all of
% these methods are not scalable to large numbers of items as they require
% learning unique parameters for every item. Specifically, such content-based
% matrix factorization methods require learning a matrix that has a row for every
% item. For items with attributes, it is often infeasible to store this matrix in
% memory or exploit efficient coordinate ascent optimization schemes that require
% processing this entire matrix.

% \subsection{Recommendation via ranking}


% We describe recommendation models trained on
% such loss functions and extensions that include side information.

% \paragraph{Learning to rank}
% The literature on learning to rank includes models that optimize proxies of
% evaluation metrics, such as mean average precision, mean reciprocal rank, or
% discounted cumulative gain~\cite{yu2018walkranker:,liang2018top-n-rank:}. Forb
% example, Bayesian personalized ranking models optimize a pairwise ranking
% objective function \cite{rendle2009bpr:} that trains the model to rank items a
% user consumed higher than items a user did not consume. This objective is a
% heuristic motivated by an analogy to the receiver operating characteristic; a
% model trained on this objective does not provably maximize this metric. 
% \citet{song2018neural} extend Bayesian personalized ranking using deep neural
% networks, but do not model side information.

% \paragraph{Learning to rank with side information}
% Models that optimize proxies of ranking metrics that use side information
% include \citet{shi2012tfmap:}, where a smoothed approximation of mean average
% precision is used as a loss function. \citet{yuan2016optimizing} use a proxy of
% a ranking loss to fit a polynomial that models predictions of item consumption
% using item and side information features. \citet{ying2016collaborative} uses
% denoising autoencoders to represent item information in a model trained with a
% pairwise ranking loss. \citet{cao2017embedding} use a ranking loss to jointly
% learn embeddings of items and attributes; they focus on the case where users
% interact directly with both attributes and items with said attributes. All of
% these models require learning unique parameters for every item, and do not scale
% to large numbers of items. An example of a scalable method that uses the
% Bayesian personalized ranking criterion is in \citet{okura2017embedding-based},
% but this approach requires data with timestamps and negative item feedback.


% \paragraph{Order-invariant models.} Deep learning architectures have been
% developed for set-valued input. Such architectures are invariant to permutations
% of set elements and can approximate any order-invariant function
% \cite{zaheer2017deep,ravanbakhsh2017equivariance}. This work
% addresses regression whereas we focus on recommendation, and develop a negative
% sampling technique.
% %
% \citet{kumar2018representation} extend the order-invariant architectures to the
% problem of a set-valued response; we focus on set-valued input for which data is
% more readily available.
% %
% \citet{benson2018sequences} study the problem of predicting sets in a sequential
% order. The task is to predict attributes of a new item given the number of
% attributes. These attributes are modeled as coming from the attributes of the
% items a user has recently consumed. In contrast, we do not focus on temporal
% data and do not focus on repeated consumption of whole or partial copies of of
% items' sets of attributes.

% \paragraph{Ranking models.} Bayesian personalized ranking models optimize a
% ranking criterion \cite{rendle2009bpr:} that trains the model to rank items a
% user consumed higher than items a user did not consume. The criterion is
% motivated by an analogy to the receiver operating characteristic, but they do
% not prove that optimizing the criterion is equivalent to optimizing this metric.
% In our work, we prove (in \Cref{prop:maximizing-recall}) that our approach
% directly optimizes recall.
% %
% The Bayesian personalized ranking criterion has been extended to recommending
% news articles \cite{okura2017embedding-based}, but this approach requires the
% collection of observed (but not consumed) items. Our method applies to data
% where this additional information is not required.

% % c.f. https://data.princeton.edu/wws509/notes/c6s3
% \paragraph{Discrete choice econometrics models.} Conditional logit models are
% used in economics to study purchasing decisions \cite{mcfadden1973conditional},
% and may include characteristics of items such as attributes.
% %
% \citet{ruiz2017shopper:} develop a sequential model for discrete choice of
% consumer behavior. They focus on predicting additional attributes for an item
% conditioned on its existing attributes, whereas our task concerns ranking items
% given their attributes.
% %
% \citet{chiong2019random} use random projections to reduce the dimensionality of
% the choice set in a discrete choice model (the number of items). However, it is
% unclear whether their model scales: they study a dataset with a choice set of
% size $3$k. The choice set in the diet data we study has tens of millions of
% items.
% %
% \citet{overgoor2018choosing} develop a discrete choice model for graph-based
% data where the task is predicting new edges. They use negative sampling as
% training data for missing links in the graph, but do not address the case where
% nodes have set-valued attributes (that is the case we focus on).

% \paragraph{Deep learning-based recommender systems.} \citet{zhang2017deep}
% reviews several deep learning models for recommending items to users. However,
% these models are recommend items without leveraging side information as we do in
% this work.
% %
% For example, \citet{nguyen2018npe:} develop a model for recommendation with
% negative sampling, where the context items are other items a user has consumed.
% (They not study the case where items are represented by sets of attributes.)
% %
% \citet{trofimov2018inferring} use a ranking loss with negative sampling for
% learning embeddings to predict attributes of an item conditional on existing
% attributes. Our task differs in that we aim to recommend items conditional on
% their full set of attributes.
% %
% \citet{chen2017joint} study the task of ranking text for users by incorporating
% different unsupervised representations of text. They do not address the task of
% recommending items that are represented by sets of attributes as we focus on
% here.

% \paragraph{Negative sampling in recommender systems.}
% \citet{chen2017on-sampling} analyze computational tradeoffs of different
% negative sampling strategies for recommender systems. Their work is
% complementary to ours, and could speed up the training of our model.

% !TEX root = recommending-interesting-writing.tex
\section{Method}
We describe the recommendation model and the pipeline for data collection, training of the model, and evaluation. This is illustrated in \Cref{fig:pipeline}.

\paragraph{Recommendation Model.} \acrfull{rfs} is a recommendation model suitable for our problem. It is scalable to large numbers of articles and words, and can maximize the evaluation metric of recall~\citep{altosaar2020rankfromsets:,altosaar2020probabilistic}. Recall, or the fraction of true positives returned by a recommendation model, is an appropriate evaluation metric for recommending interesting writing to editors at The Browser. A recommendation model such as \gls{rfs} can be readily backtested with recall as an evaluation metric, as historical data contains positive examples (articles selected by the editors) but rarely contains negative examples (articles seen but not selected by the editors).

\gls{rfs} is a recommendation model defined by a binary classifier. For a user $u$ and item $m$ with attributes $x_m$ (the set of unique words in an article), \gls{rfs} is described by the probability of $y_{um} = 1$ (user $u$ consuming item $m$):
$$p(\yum = 1 \mid u, m) = \sigma\left( f \left (u, x_m\right) \right)\, ,$$
where $\sigma$ is the sigmoid function. To parameterize the binary classifier in \gls{rfs}, we use an inner product architecture:
\begin{equation}
\label{eq:inner-product}
  f\left(u, x_m\right) = \theta_u^\top\left(\frac{1}{|x_m|}\sum_{j\in x_m}
  \beta_j\right)\, .
\end{equation}
In this architecture, the user embedding $\theta_u$ includes a dimension that is fixed to unity. Word embeddings $\beta_j$ (including a bias dimension for every word) and the publication embedding are fit with maximum likelihood estimation, and negative examples are sampled uniformly at random to balance positive examples.

\paragraph{Data Collection and Preprocessing.} For positive examples, we use the historical set of articles curated by editors at The Browser. We augment the training data with articles selected by the editors of other curation services, and treat all positively-labeled examples curated by editors as data from a single user due to a paucity of data. We use articles from news websites as examples with negative labels, and collect additional articles with negative labels from websites most-featured by the editors to mimic the editorial process of reading a large swath of articles in a feed and distilling an article list to a select few. For preprocessing the data we use the tokenizer released by \citet{devlin2019bert:} and discard words not recognized by the tokenizer. This procedure leaves a dictionary with $30$k words, and $148$k training examples with $28$k positive labels.

\paragraph{Empirical Study.} Performance was assessed with recall, and $15\%$ of the data was held-out for validation and test sets. The performance of the model was similar for large embedding sizes, and we selected the dimension of the embeddings to be $25$. We cross-validate using the RMSProp optimizer~\citep{tieleman2012lecture} and grid search over learning rates of $\{10^{-1}, 10^{-3}, 10^{-4}, 10^{-5}\}$ and select the best-performing model for deployment.

\section{Deployment and User Interface}

As curation services can be constrained by computational and financial resources, we choose to deploy the \gls{rfs} model as a cloud computing-based microservice. Further, we exploit the properties of the the \gls{rfs} architecture in \Cref{eq:inner-product} to aid explainability and exploration. For explainability, we display words $j$ with high (low) inner product $\theta_u^\top\beta_j$, as these words contribute the most (least) to a prediction of a positive label. To enable users to explore patterns in a large list of articles, we enable them to modulate the dimensions of $\theta_u$ with the greatest weight. After a user has scaled these dimensions according to their preference, the recommendation results are returned by the microservice using \Cref{eq:inner-product}.


% !TEX root = recommending-interesting-writing.tex
\section{Evaluation}
\label{sec:experiments}

We conduct an offline empirical study of the performance of \acrlong{rfs} to assess its performance as a recommendation model. Then we qualitatively evaluate the visual interface to see whether the explanation-aware, controllable interface enabled by \gls{rfs} can help make editors at The Browser make better decisions.

\paragraph{Data Collection and Preprocessing.} For positive examples, we use the historical set of articles curated by editors at The Browser. We augment the training data with articles selected by the editors of other curation services, and treat all positively-labeled examples curated by editors as data from a single user due to a paucity of data. We use articles from news websites as examples with negative labels, and collect additional articles with negative labels from websites most-featured by the editors to mimic the editorial process of reading a large swath of articles in a feed and distilling an article list to a select few. For preprocessing the data we use the tokenizer released by \citet{devlin2019bert:} and discard words not recognized by the tokenizer. This procedure results in a dictionary with $30$k words, and $646$k datapoints with $27$k positive labels.

\paragraph{Metrics} Performance of the recommendation models is assessed with recall, and $15\%$ of the datapoints are held out for the validation and test sets respectively.

\paragraph{Experimental setup: RankFromSets} We cross-validate using the RMSProp optimizer~\citep{tieleman2012lecture} with a momentum of $0.9$ and grid search over learning rates of $\{10^{-2}, 10^{-3}, 10^{-4}, 10^{-5}\}$, whether or not to initialize from pre-trained \acrshort{bert} embeddings~\citep{wolf2019huggingfaces}, and embedding sizes of $\{10,25,50, 100, 500, 1000\}$. This model is trained on an NVIDIA Tesla P100 GPU.

\paragraph{Experimental setup: \acrshort{bert}} To fine-tune \acrshort{bert}, we use the AdamW optimizer with a linear learning rate scheduler and warmup steps, with a batch size of $32$ and maximum input length of $512$ as in \citet{devlin2019bert:} and \citet{wolf2019huggingfaces}. A grid search is performed over learning rates of $\{2, 3, 4, 5\} \times 10^{-5}$, warmup steps of $\{10^2, 10^3, 10^4\}$, and total training steps of $\{10^2, 10^3, 10^4, 10^5\} \times 5$. The model is trained on an NVIDIA Tesla V100 GPU.

The best-performing model of \gls{rfs} is selected for deployment, and recall is evaluated on the test set, after using early stopping according to validation recall. The results are shown in \Cref{tab:recall}, and \gls{rfs} outperforms \acrshort{bert} by $14\%$. Furthermore, \gls{rfs} achieves better performance ten times faster than \acrshort{bert}, as shown in \Cref{fig:training-recall}. In a test to measure the speed of recommending $10^4$ held-out articles, \gls{rfs} ranked all articles in $120$ ms on a CPU, while \acrshort{bert} took $4$ m $54$ s to rank the articles on an NVIDIA Tesla V100 GPU. This represents a 2000-fold improvement in speed, which is beneficial for the controllable visual interface that requires \Cref{eq:inner-product-control} to be quickly computed in response to user input.

% !TEX root = ../recommending-interesting-writing.tex
\begin{figure}[!tb]
  \centering
  \includegraphics[width=0.95\linewidth]{fig/training-recall}
  \caption{\acrlong{rfs} achieves better performance faster than \acrshort{bert} in terms of validation recall during training.}
  \label{fig:training-recall}
\end{figure}
% !TEX root = ../recommending-interesting-writing.tex
\begin{table}[tb]
\centering
\begin{tabular}{lSS}
\toprule
Recommendation Model & \multicolumn{1}{c}{Recall @ 1000 (\%)}
\\
\midrule
\acrlong{rfs} &  \bfseries 53.1\\
\acrshort{bert} & 46.6 \\
\bottomrule
\end{tabular}
% BERT 466/1000
% rankfromsets 531/1000
% \vspace{1ex}
\caption{\gls{rfs} outperforms \acrshort{bert} in an offline evaluation, on a task of predicting which articles editors at The Browser would feature based on words in the articles.}
\label{tab:recall}
\end{table}

\paragraph{Qualitative Evaluation} In a user study, editors at The Browser provided feedback that they used the visual interface to choose articles, and found this to be an improved workflow. The control over recommendations, and explanation-aware visual interface provided by \gls{rfs} helped elicit bugs in data collection (such as foreign language sources, or fiction writing) and provides an enjoyable user experience.
% !TEX root = recommending-interesting-writing.tex
\section{Discussion}
We built a visual interface for a recommender system powered by \gls{rfs}, a flexible recommendation model. Empirically, we demonstrated that \gls{rfs} outperforms \acrshort{bert} in an offline evaluation, while being orders of magnitude faster during training and recommendation. By deploying \gls{rfs} to AWS Lambda and hosting the visual interface on Github Pages, we demonstrated a fully open-source pipeline for creating an explanation-aware, controllable visual interface for document recommendation for editorial decision-making. Future work includes studying whether the transparency and control provided by open-source recommendation systems can improve user experience and educate users in how their information is exploited remains an open problem.
\bibliographystyle{ACM-Reference-Format}
\bibliography{bib}
% !TEX root = recommending-interesting-writing.tex

\appendix
\clearpage
\section{Reproducibility supplement: code}
\label{sec:code}
We give an example implementation of \acrlong{rfs} with the inner product
regression function in \Cref{eqn:rankfromsets} in python with the PyTorch
package (version 1.0.0). This shows how easy it is to implement \gls{rfs}; we
used a similar implementation to achieve state-of-the-art results in
\Cref{sec:experiments_arxiv}.
\begin{minted}{python}
import torch
from torch import nn
import data

class InnerProduct(nn.Module):
  def __init__(
        self, n_users, n_items, n_attributes, emb_size):
    super().__init__()
    self.user_embeddings = nn.Embedding(n_users, emb_size)
    self.attribute_emb_sum = nn.EmbeddingBag(
        n_attributes, emb_size, 'mean')
    self.item_embeddings = nn.Embedding(n_items, emb_size)
    self.intercepts = nn.Embedding(n_items, 1)

  def forward(
        self, users, items, item_attributes, offsets):
    user_emb = self.user_embeddings(users)
    attr_emb = self.attribute_emb_mean(
        item_attributes, offsets)
    item_emb = self.item_embeddings(items)
    logits = user_emb * (attr_emb + item_emb)
    logits = logits.sum(-1, keepdim=True)
    return logits + self.intercepts(items)

train_data = data.load_data()
learning_rate, batch_size = 0.1, 1000
model = InnerProduct(train_data.n_users,
                     train_data.n_items,
                     train_data.n_attributes,
                     emb_size=100)
optimizer = torch.optim.SGD(
    model.parameters(), learning_rate)
loss = torch.nn.BCEWithLogitsLoss()
# negative samples are in last half of each batch
labels = torch.cat(torch.ones(batch_size // 2),
                   torch.zeros(batch_size // 2))
for batch in train_data:
  model.zero_grad()
  logits = model(*batch)
  L = loss(logits, labels)
  L.backward()
  optimizer.step()
\end{minted}

\input{table/lstm.tex}

% \section{Reproducibility supplement: permutation-invariant models}
% \label{sec:ctpf-permutation-invariance}

% \paragraph{Matrix factorization permutation-invariant.} We show that
% collaborative topic Poisson factorization~\citep{gopalan2014content-based} is
% permutation-invariant. The likelihood of the attributes (words) associated with
% items (documents) in this model factorizes, and multiplication is
% permutation-invariant. Conditional on the latent item representation $\theta_d$ and
% latent word representation $\beta_v$, every word in the document $w_{dv}$ is
% independent. The joint probability of the words in a document factorizes:
% \begin{align}
%   p(w_{d} \mid \theta_d, \beta_v) &= \prod_{w_{dv} \in w_d} p(w_{dv} \mid \theta_d, \beta_v).
% \end{align}
% In this model, predictions are made using expectations under the posterior. The
% posterior is proportional to the the log joint of the model, and the attributes
% of items (words in documents) enter into the model only via the above product.
% The product of the probability of words in a document is invariant to a
% reordering of the words in the document. Therefore, collaborative topic Poisson
% factorization is permutation-invariant.

\section{Reproducibility supplement: simulation study}
\label{sec:simulation}
% In this simulation, the regression function $f$ was the square kernel between
% the user embedding $\theta_u$ and the mean of item attribute embeddings
% ${\beta_j: j \in x^+_m}$. For a fair comparison, the regression functions in
% \Cref{eqn:rankfromsets,eqn:neural-network,eqn:residual} were chosen to have the
% same number of parameters. For data generated using the square kernel, the
% residual and deep parameterizations outperformed the inner product
% parameterization.

With finite data and a finite number of paramaters, the optimal parameterization
of \gls{rfs} is dependent on the data-generating distribution. Recall that
observations of user-item interactions are generated by a Bernoulli distribution
with logit function $f$. We describe a choice of logit function $f$ that leads
to the residual and deep architectures in \Cref{eqn:residual,eqn:neural-network}
outperforming the inner product architecture in \Cref{eqn:rankfromsets} in terms
of predictive performance. Note that the number of parameters across
architectures must be equal for a fair comparison. The code required to
replicate this experiment is included here:

%\url{https://github.com/kdd-anonymous/rankfromsets}.

We simulate data from the following generative process:

\begin{enumerate}
  \item {\bf For every user $u$:}
    \begin{enumerate}
    \item Draw user embedding $\theta_u \sim \textrm{Normal}(0, \mbI)$.
    \end{enumerate}
  \item {\bf For every item attribute $j$:}
    \begin{enumerate}
    \item Draw attribute embedding $\beta_j \sim \textrm{Normal}(0, \mbI)$.
    \end{enumerate}
  \item {\bf For every item $m$:}
    \begin{enumerate}
    \item Draw item topics $\theta_m \sim \textrm{Dirichlet}(\alpha)$
    \item Draw number of item attributes $M \sim \textrm{Poisson}(\lambda)$
    \item Draw nonzero item attributes $x_m \sim \textrm{Multinomial}(M, \theta_m)$.
    \end{enumerate}
  \item {\bf For every user, item observation:}
    \begin{enumerate}
    \item
      $\yum \sim \textrm{Bernoulli}\left(\yum; \sigma(f(\theta_u, x_m)\right)$
    \end{enumerate}
\end{enumerate}

The logit function $f$ is the square kernel:

$$f(\theta_u, x_m) = \left(\theta_u^\top \frac{1}{|x_m|}\sum_{j \in x_m}
  \beta_j\right)^2.$$

Before sampling from the Bernoulli, we standardize the logits output by $f$
across users and subtract $7$ to achieve sparse user-item observations.

\paragraph{Simulation setup} We set the Dirichlet parameter to be $\alpha=0.01$
and the Poisson rate to be $\lambda=20$. We generate data for $1$k users, $5$k
item attributes, $30$k items, and hold out $100$ users for each of the
validation and test sets. We fix the momentum to
$0.9$~\cite{sutskever2013on-the-importance} and grid search over stochastic
gradient descent learning rates of ${10, 1, 0.1, 0.01}$ and over two learning
rate decay schedules. The first linear learning rate decay goes to zero over
$100$k iterations, while the second divides the learning rate by $10$ if the
validation in-matrix recall does not improve (evaluation is performed every
$500$ iterations). We run the grid search on one instance of data generated from
this model, then for the best performing hyperparameters for each model trained
on this instance, we regenerate data $30$ times and average results over these
synthetic datasets.

\begin{table}[!h]
\sisetup{table-figures-uncertainty=1}
\centering
\begin{tabular}{
  l
  S[table-format=1.2,
   table-figures-uncertainty=1]
  S[table-format=1.2,
  table-figures-uncertainty=1]
  S[table-format=1.2,
  table-figures-uncertainty=1]
  }
  \toprule
  Regression function & {Inner product} & {Deep} & {Residual} \\
  \midrule
  Recall & 0.29 \pm 0.15 & 0.32 \pm 0.14 & 0.33 \pm 0.18 \\
  \bottomrule\\
\end{tabular}
\caption{\textbf{A simulation study demonstrating that the choice of
    parameterization of \acrlong{rfs} is data-dependent.} We report the
  in-matrix recall averaged over 100~users and 30~replications of the simulation
  where data is regenerated. The residual model in \Cref{eqn:residual}
  outperforms the deep model, \Cref{eqn:neural-network}, and the inner product
  model in \Cref{eqn:rankfromsets}.}
\label{tab:simulation}
\end{table}


%%% Local Variables:
%%% mode: latex
%%% TeX-master: "../set_recommendation"
%%% End:


\paragraph{Simulation results} The results in \Cref{tab:simulation} demonstrate
that the residual model outperforms both the deep and inner product
architectures for data generated by the above generative process.

\paragraph{Generalization} The above example shows that the choice of architecture in
\gls{rfs} is data-dependent. To ensure that the model does not
overfit as new users or items are included in the training data, we need to
compare the number of parameters to the number of datapoints. A model with
parameters the size of the training data can overfit by memorizing the training
data. For generalization to be possible, overfitting can be avoided if the
number of parameters grows slower than the size of the data. The technical
backing for this comes from asymptotic statistics and the concept of sieved
likelihoods. Specifically, the maximum likelihood estimation procedure with the
objective function in \Cref{eq:objective} can be replaced by maximization of a
sieved likelihood function. The `sieve' refers to filtering information as the
number of parameters (in this case, user and item representations) grows with
the number of observations. The sieved likelihood function enables the analysis
of asymptotic behavior as the number of users grows $U\rightarrow \infty$ and
the number of items grows $I\rightarrow \infty$, An example of a technique to
grow the number of parameters in a way that supports generalization is given in
Chapter 25 of \citet{vaart1998asymptotic}.


%%% Local Variables:
%%% mode: latex
%%% TeX-master: "set_recommendation"
%%% End:
%\clearpage
\end{document}

%%% Local Variables:
%%% mode: latex
%%% TeX-master: "set_recommendation"
%%% End:
